% Created 2018-03-06 Tue 13:27
\documentclass[11pt]{article}
\usepackage[utf8]{inputenc}
\usepackage[T1]{fontenc}
\usepackage{fixltx2e}
\usepackage{graphicx}
\usepackage{longtable}
\usepackage{float}
\usepackage{wrapfig}
\usepackage{rotating}
\usepackage[normalem]{ulem}
\usepackage{amsmath}
\usepackage{textcomp}
\usepackage{marvosym}
\usepackage{wasysym}
\usepackage{amssymb}
\usepackage{hyperref}
\tolerance=1000
\usepackage{geometry}
\geometry{a4paper}
\author{Jason Medcoff}
\date{\today}
\title{Assignment 2}
\hypersetup{
  pdfkeywords={},
  pdfsubject={},
  pdfcreator={Emacs 25.2.2 (Org mode 8.2.10)}}
\begin{document}

\maketitle
All code following was written and compiled with CLion and CMake,
including only pthread as a library.

\section{Problem 1}
\label{sec-1}

To begin, we would like to implement a threaded hash table with
chaining. To perform chaining, we first must implement a list. In pure
C, without classes, we resort to structs and pointers. The list is
implemented to be generic, using void pointers to store data.

\begin{verbatim}
/* -------------------------------- */
/* ----------- List ops ----------- */
/* -------------------------------- */

typedef struct node {
    void *data; //generic data
    struct node *next; //successor node
} node;

node* node_create(void *data, node *next) {
    node* newnode = (node*)malloc(sizeof(node));
    newnode->data = data;
    newnode->next = next;
    return newnode;
}

node* node_push(node* head, void *data) {
    node* newnode = node_create(data, head);
    head = newnode;
    return head;
}

node* node_append(node* head, void *data) {
    node* current = head;
    while (current->next != NULL)
        current = current->next;
    node* newnode = node_create(data, NULL);
    current->next = newnode;
    return head;
}

node* node_search(node* head, void *data) {
    node* current = head;
    while (current != NULL) {
        if (current->data == data)
            return current;
        current = current->next;
    }
    return NULL;
}

void node_dispose(node* head) {
    node *current, *temp;
    if (head != NULL) {
        current = head->next;
        head->next = NULL;
        while (current != NULL) {
            temp = current->next;
            free(current);
            current = temp;
        }
    }
}
\end{verbatim}

The typical list interface is found, with an additional memory
management operation, \verb~node_dispose~, to free the list.

Looking ahead to problem 2, we know we will need to place instances of
a 15-puzzle into our hash table. Thus, to simplify implementation, we
will construct the hash table specifically for the data structure
chosen for the 15-puzzle. For simplicity, a puzzle state will be
represented as a $4 \times 4$ array, containing the numbers 0
through 15. The address occupied by 0 represents the empty tile,
whereas the remaining numbers represent the tiles in order, i.e., 1
resides in the upper left corner, 4 in the upper right, 13 in the
bottom left, etc. This arrangement allows operations on puzzle states
to perform very fast, nearly indistinguishable from constant
time. This will become relevant when we are moving tiles around to
solve the game. 

In the hash table implementation, we will consider a set of entries
and a set of pthread locks, with each lock associated with a size \verb~k~
block of entries.

\begin{verbatim}
typedef struct entry {
    unsigned int key;
    node* values;
} entry;

typedef struct hashtable { // of size P
    int num_entries;
    pthread_rwlock_t locks[NLOCKS];
    //pthread_mutex_t locks[NLOCKS];
    entry** table;
} hashtable;
\end{verbatim}

To create a new hashtable, we will allocate the appropriate memory for
the entries and initialize all entry keys to zero, and all values to
\verb~NULL~. The values will eventually be a list of all states associated
with the respective keys. As with the list, we must create a routine
to free a created table from memory. However, here we must also take
care to dispose of each pthread lock we had created.

\begin{verbatim}
// init table with keys=0 and no values
hashtable* create_table() {
    entry** tab = malloc(P*sizeof(entry*));
    hashtable* newtable = malloc(sizeof(hashtable));
    newtable->table = tab;
    for (int i=0; i<P; i++) {
        newtable->table[i]->key = 0;
        newtable->table[i]->values = NULL;
    }
    for (int j=0; j<NLOCKS; j++) {
        //pthread_mutex_init(&newtable->locks[j], NULL);
        pthread_rwlock_init(&newtable->locks[j], NULL);
    }
    return newtable;
}

void free_table(hashtable* h) {
    for (int i=0; i<P; i++) {
        node_dispose(h->table[i]->values);
    }
    for (int j=0; j<NLOCKS; j++) {
        //pthread_mutex_destroy(&h->locks[j]);
        pthread_rwlock_destroy(&h->locks[j]);
    }
    free(h->table);
    free(h);
}
\end{verbatim}

Among the most challenging tasks for problem 1 is designing a somewhat
intelligent hash function for gamestates. While chaining removes the
need for extreme cleverness, a good hash function will still try to
evenly distribute values among the table. Efficiency will also be of
great importance. Here we will take the product of the first row of
the puzzle, and modulo by a number \verb~P~. Namely, the maximum possible
product is $32670 = 12\cdot 13 \cdot 14\cdot 15$, and we will modulo
by \verb~P~ = 10007, a prime number.

\begin{verbatim}
unsigned int hash_state(int a[4][4]) {
    unsigned int key = 1;
    int i;
    for (i=0; i<4; i++) {
        key *= a[0][i];
    }
    return key % P;
}
\end{verbatim}

Next, we can now design functions to add values to the table, as well
as check the table for existence of a particular value.

\begin{verbatim}
void add_to_table(hashtable* h, int a[4][4]) {
    unsigned int key = hash_state(a);
    int tn = (int) floor((key/P)*NLOCKS);
    pthread_rwlock_wrlock(&h->locks[tn]);
    //pthread_mutex_lock(&h->locks[tn]);
    h->table[key]->key = key;
    node_append(h->table[key]->values, a);
    h->num_entries++;
    pthread_rwlock_unlock(&h->locks[tn]);
    //pthread_mutex_unlock(&h->locks[tn]);
}

int check_table(hashtable* h, int a[4][4]) {
    unsigned int key = hash_state(a);
    int tn = (int) floor((key/P)*NLOCKS);
    pthread_rwlock_rdlock(&h->locks[tn]);
    //pthread_mutex_lock(&h->locks[tn]);
    node* res = node_search(h->table[key]->values, a);
    pthread_rwlock_unlock(&h->locks[tn]);
    //pthread_mutex_unlock(&h->locks[tn]);
    if (res)
        return 1;
    return 0;
}
\end{verbatim}

Note that in all code above, mutex and rwlocks are present, with one
or the other commented out to allow switching between the two used in
implementation.

For testing purposes, we want to try hashing different objects and
observe the running time. To do this, we must create objects the table
can hash. While the puzzle states will be as described above, we can
be more lenient with regards to testing, since the hash function will
operate on a 2D array by taking the product of the first row. So, our
function to create states for testing need only worry about putting
numbers in the first row of a 2D array; we will do so randomly.

\begin{verbatim}
int** rand_state() {
    // for testing purposes, we need only care about
    // the rop row
    int a[4][4];
    srand(time(NULL));
    for (int i=0; i<4; i++) {
        a[0][i] = rand() % 20; // this doesn't matter
    }
    return a;
}
\end{verbatim}

The driver for problem 1 can be written. We create 10,000 entries,
then hash them and record the time taken.

\begin{verbatim}
void driver1() {
    hashtable* ht = create_table();
    time_t now = time(NULL);
    for (int i=0; i<10000; i++) {
        add_to_table(ht, rand_state());
    }
    int n = time(NULL) - now;
    printf("time: %d\n", n);
}
\end{verbatim}



\section{Problem 2}
\label{sec-2}

As described above, puzzle states are represented as $4\times 4$
arrays holding the numbers 0 through 15 once. First, we will rigidly
define the completed game, and write a function that decides whether a
given puzzle state is finished or not. This will be used later on.

\begin{verbatim}
/* -------------------------------- */
/* ---------- Puzzle game --------- */
/* -------------------------------- */

//the finished game. 0 represents the empty tile
const int done[4][4] = {
        {1,   2,   3,  4},
        {5,   6,   7,  8},
        {9,  10,  11, 12},
        {13, 14,  15,  0}};

int is_done(int s[4][4]) {
    int i, j;
    for (i=0; i<4; i++) {
        for (j=0; j<4; j++) {
            if (s[i][j] != done[i][j])
                return 0; // not done
        }
    }
    return 1; // done
}
\end{verbatim}

Next, we will need a utility function to copy the contents of an array
to a new array. When we branch the states of the puzzle game, we will
need to create new arrays like this.

\begin{verbatim}
int** copyarray(int s[4][4]) {
    int i, j, temp[4][4];
    for (i=0; i<4; i++) {
        for (j = 0; j < 4; j++) {
            temp[i][j] = s[i][j];
        }
    }
    return (int **)temp;
}
\end{verbatim}

Next, we will need to define transition between puzzle states. The
states of the puzzle form a symmetric group under composition of
permutations, and we know that we can use products of transpositions
to reach any permutation. (That said, an important note is that not
every puzzle state can be reached from the done state, and therefore
as permutations are invertible, there exist puzzle states which are
unwinnable. In fact, exactly half of all possible layouts of a grid as
described can be won.) Thus, the only operations we need define are
four transpositions: left, right, up, and down. These designate the
motion of the empty tile 0 around the board. With successive
application of these functions, we can permute the game board as we
please.

\begin{verbatim}
int** left(int s[4][4]) {
    int i, j, temp, **ret = copyarray(s);
    for (i=0; i<4; i++) {
        for (j=0; j<4; j++) {
            if (ret[i][j] == 0 && j != 0) {
                temp = ret[i][j-1];
                ret[i][j-1] = 0;
                ret[i][j] = temp;
                return ret;
            }
        }
    }
}

int** right(int s[4][4]) {
    int i, j, temp, **ret = copyarray(s);
    for (i=0; i<4; i++) {
        for (j=0; j<4; j++) {
            if (ret[i][j] == 0 && j != 3) {
                temp = ret[i][j+1];
                ret[i][j+1] = 0;
                ret[i][j] = temp;
                return ret;
            }
        }
    }
}

int** up(int s[4][4]) {
    int i, j, temp, **ret = copyarray(s);
    for (i=0; i<4; i++) {
        for (j=0; j<4; j++) {
            if (ret[i][j] == 0 && i != 0) {
                temp = ret[i-1][j];
                ret[i-1][j] = 0;
                ret[i][j] = temp;
                return ret;
            }
        }
    }
}

int** down(int s[4][4]) {
    int i, j, temp, **ret = copyarray(s);
    for (i=0; i<4; i++) {
        for (j=0; j<4; j++) {
            if (ret[i][j] == 0 && i != 3) {
                temp = ret[i+1][j];
                ret[i+1][j] = 0;
                ret[i][j] = temp;
                return ret;
            }
        }
    }
}
\end{verbatim}

We will be storing puzzle states in a priority queue implemented as a
min-heap, so we must be able to rank states in terms of closeness to
being done. The manhattan distance, or $\ell_1$ norm, can be used to
measure the sum distance of each tile from where it is supposed to be
in the finished puzzle. Thus, we must consider coordinates of tiles,
and so we need a new data structure.

\begin{verbatim}
typedef struct coord {
    int x;
    int y;
} coord;

coord locate(int a, int s[4][4]) {
    for (int i=0; i<4; i++) {
        for (int j=0; j<4; j++) {
            if (s[i][j] == a) {
                coord res;
                res.x = i;
                res.y = j;
                return res;
            }}
        }
}
\end{verbatim}

With this, we can measure manhattan distance, and therefore come up
with a comparison function for our eventual min-heap to use.

\begin{verbatim}
int manhattan(int s[4][4]) {
    int res = 0;
    for (int a=0; a<16; a++) {
        coord c = locate(a, s);
        coord r = locate(a, done);
        res += abs(c.x - r.x) + abs(c.y - r.y);
    }
    return res;
}

bool compare(int (*s)[4][4], int (*t)[4][4]) {
    int sdist = manhattan(*s);
    int tdist = manhattan(*t);
    if (sdist < tdist)
        return true;
    else
        return false;
}
\end{verbatim}

Now, we need to be able to use a priority queue to sort puzzle
states. To save time and avoid error, we will use an already
implemented library that utilizes a heap. The library is called PQlib
and can be found at \url{https://bitbucket.org/trijezdci/pqlib}.

We will now create yet another data type, an instance. This will keep
track of puzzle states and such in the queue and hashtable. A function
to create instances is also included.

\begin{verbatim}
typedef struct instance {
    int isdone;
    int** s;
    hashtable* table;
    pq_t queue;
} instance;

instance create_instance(int s[4][4]) {
    instance it;
    it.isdone = 0;
    it.s = s;
    it.table = create_table();
    it.queue = pq_new_queue(0, &compare, NULL);
    return it;
}
\end{verbatim}

Next, we can begin solving the puzzle.

\begin{verbatim}
int subsolve(instance it) {
    add_to_table(it.table, it.s);
    pq_enqueue(it.queue, it.s, NULL);
    if (it.isdone > 0 || is_done(it.s)) {
        it.isdone += 1;
        pthread_exit(NULL);
    }
    else
        while (it.isdone<1) {
            pthread_create(NULL, NULL, pq_enqueue(it.queue, left(it.s), NULL), NULL);
            pthread_create(NULL, NULL, pq_enqueue(it.queue, right(it.s), NULL), NULL);
            pthread_create(NULL, NULL, pq_enqueue(it.queue, down(it.s), NULL), NULL);
            pthread_create(NULL, NULL, pq_enqueue(it.queue, up(it.s), NULL), NULL);
        }
}

int solve(int s[4][4]) {
    instance it = create_instance(s);
    subsolve(it);
}
\end{verbatim}

A complete code listing follows.

\begin{verbatim}
#include <stdio.h>
#include <stdlib.h>
#include <math.h>
#include <stdbool.h>
#include "PQ.h"
#include <pthread.h>
#include <sys/timeb.h>
#include <sys/random.h>

#define P 10007 // medium sized prime, used in hash
#define K 10 // size of hash table thread block
#define NLOCKS 101 // ceil(P/K)


/* -------------------------------- */
/* ----------- List ops ----------- */
/* -------------------------------- */

typedef struct node {
    void *data; //generic data
    struct node *next; //successor node
} node;

node* node_create(void *data, node *next) {
    node* newnode = (node*)malloc(sizeof(node));
    newnode->data = data;
    newnode->next = next;
    return newnode;
}

node* node_push(node* head, void *data) {
    node* newnode = node_create(data, head);
    head = newnode;
    return head;
}

node* node_append(node* head, void *data) {
    node* current = head;
    while (current->next != NULL)
        current = current->next;
    node* newnode = node_create(data, NULL);
    current->next = newnode;
    return head;
}

node* node_search(node* head, void *data) {
    node* current = head;
    while (current != NULL) {
        if (current->data == data)
            return current;
        current = current->next;
    }
    return NULL;
}

void node_dispose(node* head) {
    node *current, *temp;
    if (head != NULL) {
        current = head->next;
        head->next = NULL;
        while (current != NULL) {
            temp = current->next;
            free(current);
            current = temp;
        }
    }
}

/* -------------------------------- */
/* ---------- Hash table ---------- */
/* -------------------------------- */

typedef struct entry {
    unsigned int key;
    node* values;
} entry;

typedef struct hashtable { // of size P
    int num_entries;
    pthread_rwlock_t locks[NLOCKS];
    //pthread_mutex_t locks[NLOCKS];
    entry** table;
} hashtable;

// init table with keys=0 and no values
hashtable* create_table() {
    entry** tab = malloc(P*sizeof(entry*));
    hashtable* newtable = malloc(sizeof(hashtable));
    newtable->table = tab;
    for (int i=0; i<P; i++) {
        newtable->table[i]->key = 0;
        newtable->table[i]->values = NULL;
    }
    for (int j=0; j<NLOCKS; j++) {
        //pthread_mutex_init(&newtable->locks[j], NULL);
        pthread_rwlock_init(&newtable->locks[j], NULL);
    }
    return newtable;
}

void free_table(hashtable* h) {
    for (int i=0; i<P; i++) {
        node_dispose(h->table[i]->values);
    }
    for (int j=0; j<NLOCKS; j++) {
        //pthread_mutex_destroy(&h->locks[j]);
        pthread_rwlock_destroy(&h->locks[j]);
    }
    free(h->table);
    free(h);
}

// TODO: Come up with something intelligent
unsigned int hash_state(int a[4][4]) {
    unsigned int key = 1;
    int i;
    for (i=0; i<4; i++) {
        key *= a[0][i];
    }
    return key % P;
}

void add_to_table(hashtable* h, int a[4][4]) {
    unsigned int key = hash_state(a);
    int tn = (int) floor((key/P)*NLOCKS);
    pthread_rwlock_wrlock(&h->locks[tn]);
    //pthread_mutex_lock(&h->locks[tn]);
    h->table[key]->key = key;
    node_append(h->table[key]->values, a);
    h->num_entries++;
    pthread_rwlock_unlock(&h->locks[tn]);
    //pthread_mutex_unlock(&h->locks[tn]);
}

int check_table(hashtable* h, int a[4][4]) {
    unsigned int key = hash_state(a);
    int tn = (int) floor((key/P)*NLOCKS);
    pthread_rwlock_rdlock(&h->locks[tn]);
    //pthread_mutex_lock(&h->locks[tn]);
    node* res = node_search(h->table[key]->values, a);
    pthread_rwlock_unlock(&h->locks[tn]);
    //pthread_mutex_unlock(&h->locks[tn]);
    if (res)
        return 1;
    return 0;
}

int** rand_state() {
    // for testing purposes, we need only care about
    // the rop row
    int a[4][4];
    srand(time(NULL));
    for (int i=0; i<4; i++) {
        a[0][i] = rand() % 20; // this doesn't matter
    }
    return a;
}

void driver1() {
    hashtable* ht = create_table();
    time_t now = time(NULL);
    for (int i=0; i<10000; i++) {
        add_to_table(ht, rand_state());
    }
    int n = time(NULL) - now;
    printf("time: %d\n", n);
}



/* -------------------------------- */
/* ---------- Puzzle game --------- */
/* -------------------------------- */

//the finished game. 0 represents the empty tile
const int done[4][4] = {
        {1,   2,   3,  4},
        {5,   6,   7,  8},
        {9,  10,  11, 12},
        {13, 14,  15,  0}};

int is_done(int s[4][4]) {
    int i, j;
    for (i=0; i<4; i++) {
        for (j=0; j<4; j++) {
            if (s[i][j] != done[i][j])
                return 0; // not done
        }
    }
    return 1; // done
}

int** copyarray(int s[4][4]) {
    int i, j, temp[4][4];
    for (i=0; i<4; i++) {
        for (j = 0; j < 4; j++) {
            temp[i][j] = s[i][j];
        }
    }
    return (int **)temp;
}

//movement controls: designate motion of the empty tile

int** left(int s[4][4]) {
    int i, j, temp, **ret = copyarray(s);
    for (i=0; i<4; i++) {
        for (j=0; j<4; j++) {
            if (ret[i][j] == 0 && j != 0) {
                temp = ret[i][j-1];
                ret[i][j-1] = 0;
                ret[i][j] = temp;
                return ret;
            }
        }
    }
}

int** right(int s[4][4]) {
    int i, j, temp, **ret = copyarray(s);
    for (i=0; i<4; i++) {
        for (j=0; j<4; j++) {
            if (ret[i][j] == 0 && j != 3) {
                temp = ret[i][j+1];
                ret[i][j+1] = 0;
                ret[i][j] = temp;
                return ret;
            }
        }
    }
}

int** up(int s[4][4]) {
    int i, j, temp, **ret = copyarray(s);
    for (i=0; i<4; i++) {
        for (j=0; j<4; j++) {
            if (ret[i][j] == 0 && i != 0) {
                temp = ret[i-1][j];
                ret[i-1][j] = 0;
                ret[i][j] = temp;
                return ret;
            }
        }
    }
}

int** down(int s[4][4]) {
    int i, j, temp, **ret = copyarray(s);
    for (i=0; i<4; i++) {
        for (j=0; j<4; j++) {
            if (ret[i][j] == 0 && i != 3) {
                temp = ret[i+1][j];
                ret[i+1][j] = 0;
                ret[i][j] = temp;
                return ret;
            }
        }
    }
}

typedef struct coord {
    int x;
    int y;
} coord;

coord locate(int a, int s[4][4]) {
    for (int i=0; i<4; i++) {
        for (int j=0; j<4; j++) {
            if (s[i][j] == a) {
                coord res;
                res.x = i;
                res.y = j;
                return res;
            }}
        }
}

int manhattan(int s[4][4]) {
    int res = 0;
    for (int a=0; a<16; a++) {
        coord c = locate(a, s);
        coord r = locate(a, done);
        res += abs(c.x - r.x) + abs(c.y - r.y);
    }
    return res;
}

bool compare(int (*s)[4][4], int (*t)[4][4]) {
    int sdist = manhattan(*s);
    int tdist = manhattan(*t);
    if (sdist < tdist)
        return true;
    else
        return false;
}

typedef struct instance {
    int isdone;
    int** s;
    hashtable* table;
    pq_t queue;
} instance;

instance create_instance(int s[4][4]) {
    instance it;
    it.isdone = 0;
    it.s = s;
    it.table = create_table();
    it.queue = pq_new_queue(0, &compare, NULL);
    return it;
}

int subsolve(instance it) {
    add_to_table(it.table, it.s);
    pq_enqueue(it.queue, it.s, NULL);
    if (it.isdone > 0 || is_done(it.s)) {
        it.isdone += 1;
        pthread_exit(NULL);
    }
    else
        while (it.isdone<1) {
            pthread_create(NULL, NULL, pq_enqueue(it.queue, left(it.s), NULL), NULL);
            pthread_create(NULL, NULL, pq_enqueue(it.queue, right(it.s), NULL), NULL);
            pthread_create(NULL, NULL, pq_enqueue(it.queue, down(it.s), NULL), NULL);
            pthread_create(NULL, NULL, pq_enqueue(it.queue, up(it.s), NULL), NULL);
        }

}

int solve(int s[4][4]) {
    instance it = create_instance(s);
    subsolve(it);
}



void solve_2(int s[4][4]) {
    instance it = create_instance(s);
    int finished = 0;
    while (!finished) {
        pthread_create()
    }
}
\end{verbatim}
% Emacs 25.2.2 (Org mode 8.2.10)
\end{document}